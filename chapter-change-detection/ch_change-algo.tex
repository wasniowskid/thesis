\section{Szczegóły implementacyjne}
Jednym z celów tej pracy była próba zastosowania współczesnych platform do analizy strumieniowej (rozdział \ref{ch:tech})
do rozwiązania problemu wykrywania zmian.
Zastosowanie metod opartych na wartościach progowych nie należy do zbyt skomplikowanych zagadnień
i już dziś jest wiele gotowych rozwiązań.
Dlatego w dalszej części pracy skupiono się na metodach analitycznych.

Obecnie istnieje wiele gotowych algorytmów wykrywających zmianę,
jednak wymagają korzystania ze specjalistycznych plaform (Matlab) albo nie mają wersji
przystosowanych do analizy strumieniowej.
Jednym z takich algorytmów jest algorytm Bayesa (Adams i MacKay, 2007).
Należy do grupy algorytmów przetwarzających dane na bieżąco \textit{online},
dzięki czemu łatwo było wykonać port na platformę Storm.

Innym popularnym algorytmem jest ADWIN (Bifet i in., 2011).
Należy do platformy MOA (\textit{Massive Online Analysis}).
Oryginalny algorytm, dostarczony przez autorów,
wykorzystuje relatywnie prosty test na sprawdzenie czy zmiana zaszła.
Dokładny opis znajduję się w następnym rozdziale.
Niestety test zaproponowany przez Bifeta (Bifet i in., 2011) nie działa skutecznie,
gdy problem okazuje się bardziej skomplikowany -- na przykład dodatkowe wymiary.
Dlatego dodatkowym celem postawionym przed pracą,
była próba poprawienia tego algorytmu.
Jako sposób osiągnięcia go wybrano zaproponowanie i zaimplementowanie innego testu --
opartego na stosunku funkcji gęstości rozkładów prawdopodobieństwa.
