\chapter{Zakończenie}
W ramach pracy magisterskiej została przedstawione zagadnienie wykrywania zmian w kontekscie analizy strumieniowej.
Dokonano przeglądowej analizy jakościowej i wydajnościowej wybranych platform z rozdziału \ref{ch:tech}.
Z wykorzytniem Apache Storm zostały zaimplementowane algorytmy z rodziału \ref{ch:algo},
dzięki czemu możliwe bylo sprawdzenie ich możliwości.

Metody wykrywania zmian opisane w niniejszej pracy,
mogą mieć wiele praktycznych zastosowań.
Utworzone modele wzbogacone o funkcji związane z interfejsem użytkownika (GUI) mogą posłużyć
jako moduły systemów monitorujących, alaramowych, e-commerce i wielu innych.

W rodziale \ref{ch:changes} przedstawiono wybrane algorytmy wykrywania zmian realizujące jeden z celów pracy.
Wyniki przestawione w rozdziale piątym pokazują,
że wszystkie sprawdzone metody wykrywają zmiany na zadowalającym poziomie.
Zasadnicze różnice pojawiają się w zużyciu zasobów i czasach potrzebnych do otrzymania wyników.
Różnice wynikają w wewnętrznej budowy algorytmów.
Algorytmem o największym potencjale jeśli chodzi o wyniki jest $ADWIN_d$,
jednak jest także tym, który działał najwolniej.
Z uwagi na małe różnice pomiędzy $ADWIN_d$, a $ADWIN_\mu$ dla danych jednowymiarowych
wydaje się że optymalnym zastowaniem do celów praktyczych jest $ADWIN_\mu$.

Na podstawie wyników można stwierdzić,
że wadą wszystkich metod jest wysoka częstotliwość generowania fałszywych alarmów.
Tak wysokie ich występowanie podważa sens praktycznego zastowania analizowanych algorytmów.
Dlatego też dalszym kierunkiem rozwoju pracy może być próba ustabilizowania danych wejściowych
poprzez zastosowanie różnego rodzaju filtrów wygładzających.
Innym możliwym kierunkiem jest wybór mniej złożonych testów na zgodność rozkładów.
