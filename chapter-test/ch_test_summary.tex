Wyniki przeprowadzonych testów pokazują,
że możliwe jest wykorzystanie współczesnych platform do analizy strumieniowej
do problemu wykrywania zmian.
Wszystkie algorytmy z rozdziału \ref{ch:algo} osiągneły skuteczność w wykrywaniu zmian
na poziomie około 30\% do 50\%.
Metody oparte o okno przesuwne (ADWIN) okazały się jednak skuteczniejsze.

W przypadku danych jednowymiarowych otrzymane wyniki dla rodziny ADWIN były bardzo do siebie zbliżone.
Rożnica pomiędzy nimi pojawiła się dopiero podczas testów na danych wielowymiarowych.
Z uwagi na budowę swojego testu $\mbox{ADWIN}_\mu$, opartego na średniej i wariancji,
nie był w stanie wykrywać zmian.
$\mbox{ADWIN}_d$ oparty na generycznych teście ilorazu gęstości rozkładu prawdopodobieństw radził sobie bez problemów.
Niestety obarczony był dużym narzutem wydajnościowym,
związanym z koniecznością rozwiązania zadania optymalizacyjnego podczas obliczania estymaty ilorazu.

Wszystkie badane algorytmy cechowały się dużym współczynnikem generowanych fałszywych alaramów.
Najprawdopodobniej wynika to z braku wcześniejsze pre-procesowania danych wejściowych.
Jako pre-processing można rozumieć na przykład wygładzenie przebiegów poprzez użycie róznego rodzaju filtrów.

Przeprowadzone eksperymenty pokazują,
że nie ma algorytmu idealnego (uniwersalnego),
pasującego do wszystkich kategorii problemów.
Każdy z nich pasuje do innych sytuacji.
$\mbox{ADWIN}_\mu$ nie obsługuje danych wielowymiarowych,
ale za to skutecznie wykrywa zmiany w środowisku jednowymiarowym.
Dodatkowo charakteryzuje się małym zapotrzebowaniem na zasoby i bardzo szybkimi odpowiedziami.
$\mbox{ADWIN}_\mu$ efektywnie wykrywa zmiany w rozkładach jedno i wielowymiarowych,
jednak charakteryzuje się dużą złożonością i wysokimi czasami odpowiedzi,
co niestety nie pozwala na jego wykorzystanie praktyczne.
Algorytm Bayesa także wykrywa zmiany w rozkładach jedno i wielowymiarowych.
Jego mankamentem jest konieczność wcześniejszego znania rozkładu jakim opisane są próbki,
przez co nie można go zastosować do rozwiązywania generycznych problemów.
