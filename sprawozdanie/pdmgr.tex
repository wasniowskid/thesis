\documentclass[a4paper,12pt]{scrartcl}
\usepackage{vmargin} %This give you full control over the used page arae, it maybe not the idea od Latex to do so, but I wanted to reduce to amount of white space on the page
\setpapersize{A4}
\setmargins{3.5cm}%			%linker Rand, left edge
					 {1.5cm}%     %oberer Rand, top edge
           {14.7cm}%		%Textbreite, text width
           {23.42cm}%   %Texthoehe, text hight
           {14pt}%			%Kopfzeilenh�he, header hight
           {1cm}%   	  %Kopfzeilenabstand, header distance
           {0pt}%				%Fu�zeilenhoehe footer hight
           {2cm}%

 \usepackage{t1enc} % as usual
 \usepackage[utf8]{inputenc}
 \usepackage[OT4]{polski}
 \usepackage{times}
 \usepackage{color} %allows to mark some entries in the tables with color
 \usepackage{eso-pic} % these two are required to add the little picture on top of every page
 \usepackage{everyshi} % these two are required to add the little picture on top of every page

 \renewcommand{\floatpagefraction}{0.7} %default:0.5 allows two big pictures on one page
 % ********* Table layout **************
 \usepackage{booktabs}	  	%design of table, has an excellent documentation
 %\usepackage{lscape}			%use this if you want to rotate the table together with the lines around the table

 % ********* Caption Layout ************
 \usepackage{ccaption} % allows special formating of the captions
 \captionnamefont{\bf\footnotesize\sffamily} % defines the font of the caption name (e.g. Figure: or Table:)
 \captiontitlefont{\footnotesize\sffamily} % defines the font of the caption text (same as above, but not bold)
 \setlength{\abovecaptionskip}{0mm} %lowers the distace of captions to the figure


 % ********* Header and Footer **********
 % This is something to play with forever. I use here the advanced settings of the KOMA script

 \usepackage{scrpage} %header and footer using the options for the KOMA script
 \renewcommand{\headfont}{\footnotesize\sffamily} % font for the header
 \renewcommand{\pnumfont}{\footnotesize\sffamily} % font for the pagenumbers

 %the following lines define the pagestyle for the main document
 \defpagestyle{cb}{%
 (\textwidth,0pt)% sets the border line above the header
 {\pagemark\hfill\headmark\hfill}% doublesided, left page
 {\hfill\headmark\hfill\pagemark}% doublesided, right page
 {\hfill\headmark\hfill\pagemark}%  onesided
 (\textwidth,1pt)}% sets the border line below the header
 %
 {(\textwidth,1pt)% sets the border line above the footer
 {{\it Sprawozdanie} \textsl{PDMGR}\hfill \textit{Dominik Waśniowski}}% doublesided, left page
 {\textit{Dominik Waśniowski}\hfill{\it Sprawozdanie} \textsl{PDMGR}}% doublesided, right page
 {\textit{Dominik Waśniowski}\hfill{\it Sprawozdanie} \textsl{PDMGR}} % one sided printing
 (\textwidth,0pt)% sets the border line below the footer
 }

 %this defines the page style for the first pages: all empty
 \renewpagestyle{plain}%
 	{(\textwidth,0pt)%
 		{\hfill}{\hfill}{\hfill}%
 	(\textwidth,0pt)}%
 	{(\textwidth,0pt)%
 		{\hfill}{\hfill}{\hfill}%
 	(\textwidth,0pt)}

 %********** Footnotes **********
 \renewcommand{\footnoterule}{\rule{5cm}{0.2mm} \vspace{0.3cm}} %increases the distance of footnotes from the text
 \deffootnote[1em]{1em}{1em}{\textsuperscript{\normalfont\thefootnotemark}} %some moe formattion on footnotes

 %################ End Preferences, Begin Document #####################

 \pagestyle{plain}
 \renewcommand{\labelenumi}{\arabic{enumi} }
 \renewcommand{\labelenumii}{\arabic{enumi}.\arabic{enumii} }
 \renewcommand{\labelenumiii}{\arabic{enumi}.\arabic{enumii}.\arabic{enumiii}}

 \begin{document}
 \date{}
 \begin{titlepage}
 \begin{center}

 \textsc{\LARGE Politechnika Warszawska}\\[1.5cm]

 \textsc{\Large Sprawozdanie z Pracowni dyplomowej magisterskiej}\\[1.5cm]


 % Title
 %\HRule \\[0.4cm]
 {\huge Analiza strumieniowa w Internecie Rzeczy}\\[3.9cm]



 % Author and supervisor
 \begin{minipage}{0.4\textwidth}
 \begin{flushleft} \large
 \emph{Autor:}\\
 Dominik \textsc{Waśniowski}
 \end{flushleft}
 \end{minipage}
 \begin{minipage}{0.4\textwidth}
 \begin{flushright} \large
 \emph{Opiekun naukowy:} \\
 Dr inż. Janusz \textsc{Granat}
 \end{flushright}
 \end{minipage}

 \vfill

 % Bottom of the page


 \end{center}

 \end{titlepage}
 \newpage
 \pagestyle{cb} % now we want to have headers and footers
 \begin{abstract}
 \textbf{\large
 \begin{center}
 Streszczenie
 \end{center}}
 Tematem pracy jest próba wykorzystania technik analizy strumieniowej do wykrywania zmian
 i sytuacji nietypowych w elementach układów związanych z Internetem Rzeczy.
 Układy takie generują ogromne ilości danych,
 przy których tradycyjne metody przetwarzania okazują się nieskuteczne.
 W ramach pracy zostały sprawdzone dostępne na rynku platformy do analizy strumieniowej pod kątem
 przydatności.
 Na docelowej platformie zaimplementowano algorytmy wykrywania zmian oparte na metodach statystycznych.
 Testy przeprowadzono z wykorzystaniem rzeczywistych danych wygenerowanych przez czujniki,
 jak i wygenerowanych losowo.
 \end{abstract}
 \newpage

 \section*{Spis treści pracy}
 \begin{enumerate}
	 \item \textbf{Wprowadzenie}
	 \begin{enumerate}
	 		\item Sformuowanie zadania \dotfill 1
			\item Cel i zakres pracy \dotfill 2
	 \end{enumerate}
	 \item \textbf{Problem analizy danych strumieniowych}
	 \begin{enumerate}
	 	\item Przetwarzanie dużej ilości danych - Big Data \dotfill 3
		\item Analiza strumieniowa \dotfill 8
	 \end{enumerate}
	 \item \textbf{Dostępne platformy}
	 \begin{enumerate}
	 	\item Esper \dotfill 11
		\item Apache Spark \dotfill 15
		\item Apache Storm \dotfill 19
	 \end{enumerate}
	 \item \textbf{Algorytmy}
	 \begin{enumerate}
	 	\item Algorytm Bayesa \dotfill 24
		\item Algorytmy z pakietu MOA \dotfill 26
	 \end{enumerate}
	 \item \textbf{Badania i analiza eksperymentalna}
	 \begin{enumerate}
	 	\item Opis testów \dotfill 31
		\item Wynik testów \dotfill 33
		\item Podsumowanie \dotfill 40
	 \end{enumerate}
	 \item \textbf{Podsumowanie}
 \end{enumerate}
 \newpage
 \section*{Raport pracy}
	\begin{enumerate}
		\item Pogłębienie analizy literatury dotyczącej Internetu Rzeczy i analizy strumieniowej.
		\item Analiza dostępnych platform do przetwarzania strumieniowego. Sprawdzono:
		\begin{itemize}
			\item Esper
			\item Apache Spark
			\item Apache Storm
		\end{itemize}
		Aby sprawdzić działanie powyższych platform zaimplementowano to samo zadanie polegające
		na wykrywaniu nieprawidłowości w sieci telefonicznej.
		\item Implementacja z wykorzystaniem Apache Storm algorytmu Bayesa do wykrywania zmian.
		\item Testowanie poprawności rozwiązania z wykorzytaniem danych wygenerowanych losowo.

	\end{enumerate}
 \end{document}
