\subsection{Esper}
Esper firmy EsperTech Inc. jest silnikiem umożliwiającym złożoną analizę zdarzeń
(\textit{Complex Event Processing, CEP}).
możliwym do osadzenia w dowolnej aplikcji Java lub .Net.
Esper dzięki swojemu bardzo ekspresywnemu językowi umożliwia na swobodne manipulowanie strumieniami danych.

Język Esper to specjalnie przygotowany język domenowy (\textit{domain specific language, DSL}) do obsługi zdarzeń.
Język ma formę zapytań (kwerend) i został oparty na \textit{SQL}.

\begin{lstlisting}[captionpos=b, caption=Przykładowe zapytanie w języku Esper]
  select * from StockTickEvent(symbol='IBM').win:length(100)
\end{lstlisting}

Niewątpliwie jedną z głównych zalet Espera jest jego prostota instalacji.
Wystarczy dołączyć bibliotekę,
która jest w formie pojedyńczego pliku \textit{jar}
do aplikacji i rozwiązanie jest gotowe do użytku.
Kolejną zaletą jest język dający szerokie pole do działania.
Dzięki oparciu go o klasyczny \textit{SQL} język ten jest zrozumiały dla osób dopiero rozpoczynających naukę,
jak i dla osób bez podstaw informatycznych.
Dostępne funkcje to:
filtrowanie wyników
\begin{lstlisting}
  select * from
    StockTickEvent
  where symbol='IBM'
\end{lstlisting}
funkcje agregujace
\begin{lstlisting}
  select symbol, avg(value) as value from
    StockTickEvent
  group by symbol
\end{lstlisting}
funkcje grupujące (\textit{window functions})
\begin{lstlisting}
  select * from
    StockTickEvent.win:length(100)
\end{lstlisting}
łączenie strumieni
\begin{lstlisting}
  select * from
    TickEvent.std:unique(symbol) as t,
    NewsEvent.std:unique(symbol) as n
  where t.symbol = n.symbol
\end{lstlisting}
tworzenie nowych strumieni
\begin{lstlisting}
  insert into IbmStockTickEvent
  select * from
    StockTickEvent
  where symbol='IBM'
\end{lstlisting}

Niestety to co okazało się zaletą Espera czyli prostota instalacji
jest także jego wadą.
Esper nie ma żadnych wbudowanych mechanizmów ułatwiających skalowanie,
tj. jest tak samo skalowalny jak aplikacja w której został osadzony.
