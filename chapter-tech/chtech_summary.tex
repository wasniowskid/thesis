\section{Podsumowanie}

Wszystkie opisane frameworki służą do obsługi przetwarzania strumieniowego.
Mimo że służą do podobnych zastosowań,
pomiędzy nimi istnieją zasadnicze różnice.
Zarówno w obsłudze danych,
jak i w podejściu do samego przetwarzania.

W ramach niniejszej pracy wcześniej wymienione platformy zostały sprawdzone pod kątem:
\begin{itemize}
  \item możliwości jakie oferują,
  \item sposobu korzystania zarówno od strony użytkownika jak i programisty.
\end{itemize}

\subsubsection*{Sposób procesowania, opóźnienia, skalowalność}
Największą różnicą pomiędzy omawianymi platformami jest tryb w jakim procesowane są zdarzenia.
Spark buforuje (zbiera) napływające dane w zadanym interwale czasu i dopiero potem procesuje,
natomiast procesuje dane bezpośrednio po ich otrzymaniu.
Esper dzięki swoim wbudowanym mechanizmom może pracować w obydwu trybach.

Różne tryby przetwarzania mają swoje konsekwencje w opóźnieniach (\textit{latency}).
Spark z uwagi na występowanie okien czasowych ma je zauważalne,
z kolei Storm przetwarzający dane od razu ma opóźnienia praktycznie pomijalnie małe.

Sposób przetwarzania ma także wpływ na gwarancje jakie oferują frameworki.
Ponieważ Spark nie musi pilnować pojedynczo danych,
tylko mini-batche może zagwarantować,
że dane zdarzenie przeprocesuje się dokładnie raz (\textit{exacly once}).
Storm natomiast śledzi wszystkie rekordy indywidualnie jak poruszają się przez system,
dlatego jedyne gwarancje jakie daje,
to że dane zostaną przeprocesowane co najmniej raz (\textit{at least once}).
Powoduje to, że mogą pojawić się duplikaty - niektóre rekordy mogą być przeprocesowane więcej niż jeden raz,
na przykład podczas awarii węzła.
Esper natomiast jest tylko silnikiem osadzanym wewnątrz innych aplikacji,
nie ma wbudowanych mechanizmów zarządzania klastrami,
dlatego problem zachowania gwarancji nie istnieje.
\begin{table}[h]
  \label{tab:ModelComparison}
  \begin{tabular}{p{0.2\linewidth} | p{0.25\linewidth} | p{0.25\linewidth} | p{0.2\linewidth}}
    & Esper & Spark & Storm \\
    \hline
    typ & silnik CEP & ogólny & ogólny \\
    \hline
    model & obydwa & buforowanie & niezależne \\
    \hline
    gwarancje & co najwyżej raz & co najwyżej raz & co najmniej raz \\
    \hline
    skalowalność & nie & tak & tak \\
  \end{tabular}
  \caption{Porównanie platform}
\end{table}

\subsubsection*{Implementacja, API programistyczne}
Storm w większości został napisany w języku Clojure.
API programistyczne jest przygotowane dla javy,
jednak można wykorzystywać je we wszystkich językach wykorzystujących wirtualną maszynę (\textit{jvm}).
Dodatkowo elementy Storma może wykonywać skrypty napisane w innych językach: python, R, etc.
Spark został napisany w scali,
posiada przygotowane API w javie, scali i R.
Esper powstał w języku java, chociaż posiada port napisany w języku C\#.
Do wykonywania operacji na danych wykorzystuje własny język oparty na SQL.
\begin{table}[h]
  \label{tab:ProgrammingApi}
  \begin{tabular}{p{0.2\linewidth} | p{0.25\linewidth} | p{0.25\linewidth} | p{0.2\linewidth}}
    & Esper & Spark & Storm \\
    \hline
    napisany w & java & scala & clojure \\
    \hline
    api & java, C\# & jvm, python, R & wiele \\
  \end{tabular}
  \caption{Porównanie platform -- API}
\end{table}

\subsubsection*{Instalacja, użytkowanie}
Najprostszym sposobem instalacji cechuje się Esper.
Biblioteka ma formę pojedynczego pliku, który łatwo podpiąć pod istniejące aplikacje.
System storm posiada dwa tryby lokalny i zdalny.
Tryb lokalny pozwala na tworzenie topologii na komputerach bez konieczności tworzenia klastrów.
Dzięki temu programiści mają ułatwione możliwości rozwoju aplikacji czy szukania przyczyn błędów.
Tryb zdalny jest przeznaczony dla klastrów.
Topologie w formie archiwum przesyła się do nadzorcy klastra,
a on już rozpropagowuje ją wewnątrz.
Apache Spark jest niestety pozbawiano trybu lokalnego.
Trzeba stworzyć klaster składający się z jednego węzła.

Esper charakteryzuje się także najprostszym sposobem operowania na strumieniach.
Dostarczony własny język jest bardzo ekspresywny i bogaty w funkcjonalności.
W przypadku Apache Spark dostęp do danych jest ukryty za pomocą wewnętrznych struktur.
Dzięki takiemu podejściu na start dostępnych jest wiele funkcjonalności,
jednak próba utworzenia dodatkowych elementów nie zawsze może skończyć się sukcesem.
Brak trybu lokalnego nie ułatwia tworzenia testowych źródeł danych.
Apache Storm jest najbardziej generycznym narzędziem.
Większość elementów trzeba wykonać samodzielnie,
przez co próg wejścia jest większy niż w przypadku pozostałych frameworków.
\begin{table}[h]
  \label{tab:ProgrammingLevel}
  \begin{tabular}{p{0.2\linewidth} | p{0.25\linewidth} | p{0.25\linewidth} | p{0.2\linewidth}}
    & Esper & Spark & Storm \\
    \hline
    instalacja & pojedynczy jar & cały projekt & tryb lokalny/zdalny \\
    \hline
    użytkowanie & własny DSL & wewnętrzne api & wewnętrzne api i dowolne skrypty
  \end{tabular}
  \caption{Porównanie platform -- instalacja}
\end{table}
