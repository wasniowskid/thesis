Szacuje się,
że obecnie w użyciu jest 2 mld komputerów,
10 mld telefonów komórkowych,
a do 2020 r. do sieci ma być podłączone 20 mld urządzeń (Gartner, Inc., 2015).
Urządzenia te będą zbierać dane,
przetwarzać je
bądź przekazywać dalej.
Dzięki temu ludzie będą mogli wiedzieć więcej na temat otaczającego świata,
szybciej reagować na zmiany
czy podejmować lepsze decyzje.
Ogromne ilości danych generowane w każdej sekundzie,
oraz wymagania użytkowników aby wyniki otrzymywać coraz szybciej,
najlepiej od razu,
powodują,
że obecne metody przetwarzania danych stają się niewystarczające.

Jednym z zagadnień związanych z przetwarzaniem danych jest wykrywanie zmian (sytuacji nietypowych).
Przez lata powstało wiele prac starających się rozwiązać ten problem.
Większość z nich oparta jest na całościowej analizie,
tj. zestawy danych poddanych analizie są przygotowane wcześniej i nie ulegają zmianie w czasie.
Nie są one jednak w stanie na bieżąco analizować napływających danych.

Tematem niniejszej pracy jest jeden ze sposobów wykrywania zmian,
wykorzystujący mechanizmy statystyczne.
Celem pracy jest stworzenie mechanizmów wykrywających sytuacje nietypowe
z wykorzystaniem technik analizy strumieniowej.
W ramach pracy zostaną sprawdzone dostępne platformy streamingowe pod kątem przydatności.
Zostaną zaimplementowane algorytmy Bayesa (\textit{Bayesian Online Changepoint Detection})
oraz ADWIN.
Zostanie także przeprowadzona analiza skuteczności badanych algorytmów.

Układ pracy jest następujący.
Rozdział drugi przybliża tematykę związaną z analizą strumieniową.
W tym samym rozdziale znajduję się także porównanie dostępnych platform.
W rozdziale trzecim przedstawiono teoretyczne aspekty związane z wykrywaniem zmian.
Rozdział czwarty z kolei dokładnie przedstawia wykorzystane algorytmy.
W rozdziale piątym znajdują się wyniki przeprowadzonych eksperymentów.
Ostatni rozdział stanowi podsumowanie.
W dodatku A znajduje spis zawartości płyty CD.
