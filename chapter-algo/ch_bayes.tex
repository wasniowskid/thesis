\section{Algorytm Bayesa}
\label{sec:Bayes}
Zaproponowany przez Adamsa (Adams i MacKay, 2007) algorytm Bayesa (\textit{Bayesian Online Changepoint Detection})
oparty jest na wykorzystaniu prawdopodobieństwa warunkowego Bayesa.
Algorytm dzięki swoim właściwościom możliwy jest do zastosowania w analizie strumieniowej,
ponieważ nie potrzebuje kompletu danych by wskazać miejsca wystąpienia zmiany.
Zainteresowany jest wyłącznie ostatnią.

Głównym elementem algorytmu jest przebieg (\textit{run, r}).
Przebieg $t$ jest to zbiór napływających próbek od ostatniej zmiany do chwili $t$.
Długość przebiegu to liczba próbek należąca do niego.
W chwili nadejścia nowej próbki,
możliwe są dwie opcje.
Nowa próbka należy do aktualnego przebiegu i jest do niego dodawana,
albo nastąpiła zmiana i przebieg jest zerowany.
Dzięki temu i na podstawie danych zebranych w aktualnym przebiegu
możliwe jest przewidzenie prawdopodobieństwa wystąpienia zmiany w chwili $t$,
$P(r_t|x_{1:t})$.

\subsection*{Kroki algorytmu}
\begin{enumerate}
  \item \textbf{Inicjalizacja} -- ustalenie wartości początkowych zgodnych z założonym rozkładem a priori.
  \item \textbf{Pobranie nowej próbki}.
  \item \textbf{Aktualizacja prawdopodobieństw}:
    \begin{itemize}
      \item wzrostu przebiegu,
      \item zmiany.
    \end{itemize}
  \item \textbf{Wyznaczenie nowej wartości przebiegu}.
  \item \textbf{Aktualizacja} -- modyfikaja parametrów algorytmu.
\end{enumerate}
